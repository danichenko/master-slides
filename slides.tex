%----------------------------------------------------------------------------------------
%	PACKAGES AND THEMES
%----------------------------------------------------------------------------------------

\documentclass{beamer}

\mode<presentation> {
\usetheme{Goettingen}
\setbeamertemplate{footline}[page number] 
\setbeamertemplate{navigation symbols}{}
}


\usepackage[russian]{babel}
\usepackage[utf8]{inputenc}

\usepackage{graphicx}

%----------------------------------------------------------------------------------------
%	TITLE PAGE
%----------------------------------------------------------------------------------------

\title{Построение автоматной модели по данным трассировки программы}

\author[]{Андрей Данильченко \\
Научный руководитель: Царев Ф. Н.} % Your name
\institute[НИУ ИТМО]{СПб НИУ ИТМО}
\date{16 мая 2013}

\begin{document}

\begin{frame}
\titlepage
\end{frame}

%----------------------------------------------------------------------------------------
%	PRESENTATION SLIDES
%----------------------------------------------------------------------------------------

%------------------------------------------------
\section{Введение} 
\subsection{Постановка задачи}

%------------------------------------------------

\begin{frame}
\frametitle{Мир управляющих автоматов}
\begin{itemize}
\item Понятные модели
\item Можно автоматически строить тесты
\item Легко проводить верификацию
\end{itemize}
\end{frame}

%------------------------------------------------

\begin{frame}
\frametitle{Реальность}
\begin{itemize}
\item Есть только бинарные файлы
\item Нет ясной модели поведения программы
\item Трудно строить тесты
\item Не подлежит верификации напрямую
\end{itemize}
\end{frame}

%------------------------------------------------

\subsection{Существующие решения}

\begin{frame}
\frametitle{\subsecname~---~PACHIKA}
V. Dallmeyer, A. Zeller, B. Meyer, 2009 

\begin{itemize}
\item Строит ДКА по трассировкам
\item Склеивает состояния автомата на основе \textit{упрощенных} значений полей:
\begin{align*}
boolean & \rightarrow x \mbox{ or } !x\\
numeric &\rightarrow x \ge 0 \mbox{ or }  x < 0\\
Object &\rightarrow x == null \mbox{ or } x != null
\end{align*}
\end{itemize}
\end{frame}

%------------------------------------------------

\begin{frame}
\frametitle{\subsecname~---~GK-Tail}
D. Lorenzoli, L. Mariani, M. Pezz\`{e}, 2006-2008

\begin{itemize}
\item Строит \textit{расширенный} ДКА по трассировкам
\item Склеивает одинаковые последовательности вызовов с разными параметрами
\item Склеивает состояния, если $k$-хвост одинаковый
\end{itemize}
\end{frame}

%------------------------------------------------

\begin{frame}
\Huge{\centerline{Спасибо!}}
\end{frame}

%----------------------------------------------------------------------------------------

\end{document} 